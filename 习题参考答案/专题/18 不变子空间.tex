\phantomsection
\section*{18 不变子空间}
\addcontentsline{toc}{section}{18 不变子空间}

\vspace{2ex}

\centerline{\heiti A组}
\begin{enumerate}
    \item \begin{enumerate}
              \item $\forall\alpha\in\ker S$,有$S\alpha=0$,则$T(S\alpha)=S(T\alpha)=0$,因此$T\alpha\in\ker S$,即$\ker S$是$T$的不变子空间;

              \item $\forall\alpha\in\im S$,则$\exists\beta\in V, \alpha=S\beta$,则$T\alpha=T(S\beta)=S(T\beta)$,因此$T\alpha\in\im S$,即$\im S$是$T$的不变子空间.
          \end{enumerate}

    \item 由题意有$T(e_1)=2e_1$,$Te_2=e_1+2e_2$.
          \begin{enumerate}
              \item $W_1=\spa(e_1)$,而$T(e_1)=2e_1\in W_1$,故结论成立;

              \item 反证法;设$\mathbf{R}^2=W_1\oplus W_2=\spa(e_1)\oplus W_2$,则$W_2$也是$T$的一维不变子空间,设$W_2=\spa(\alpha)$,由于$\alpha\in V$,可设$\alpha=k_1e_1+k_2e_2,\enspace k_1,k_2\in\mathbf{R}$,且由直和可知$k_2\neq 0$. 设$T\alpha=l\alpha,\enspace l\in\mathbf{R}$,即$T\alpha=T(k_1e_1+k_2e_2)=(2k_1+k_2)e_1+2k_2e_2=l(k_1e_1+k_2e_2)$,比较系数得$k_2=0$,矛盾!
          \end{enumerate}

    \item \begin{enumerate}
              \item 对于$(x,0)\in U$,有$T(x,0)=(0,0)\in U$,故$U$是$T$的不变子空间,且$T\vert_U$是零变换;

              \item 假设存在,则$\dim W=1$,即$W$是一维不变子空间,则$W$中每个非零向量都是$T$的特征向量,但我们很容易求得$T$的特征值只有0,且对应的特征向量都在$U$中,故矛盾(或与上一大题的(2)使用类似方法);

              \item 对于$(x,y)\in\mathbf{F}^2$有$(T/U)((x,y)+U)=T(x,y)+U=(y,0)+U=0+U$,故$T/U$是$\mathbf{F}^2/U$上的零变换.
          \end{enumerate}

    \item 由于$\dim\im T=\dim V-\dim\ker T$,原式等价于
          \[ \dim E(\lambda_1,T)+\cdots+\dim E(\lambda_m,T)+\dim E(0,T)\leqslant\dim V, \]
          根据$\lambda_1,\ldots,\lambda_m$非零互异,以及代数重数大于等于几何重数,可知不等式成立.

    \item 若$k=\dim V$,由于代数重数大于等于几何重数,故$T$最多有$k$个特征值;若$k<\dim V$,则$T$最多有$k$个非零特征值,再加上0,故$T$最多有$k+1$个特征值.

    \item 根据正文中讨论的,$\sigma$的特征值与$A$的一致,因此只需求解特征多项式$f(\lambda)=|\lambda E-A|=\begin{vmatrix}
                  \lambda-1 & -2        & -2        \\
                  -2        & \lambda-1 & -2        \\
                  -2        & -2        & \lambda-1
              \end{vmatrix}=(\lambda+1)^2(\lambda-5)=0$,解得特征值为$-1$(二重)和5. 解方程$AX=-X$,即$(-E-A)X=0$得基础解系为$\alpha_1=(1,0,-1)^T$,$\alpha_2=(0,1,-1)^T$,这是关于特征值$-1$的两个线性无关特征向量. 解方程$AX=5X$,即$(5E-A)X=0$得基础解系为$\alpha_3=(1,1,1)^T$,这是关于特征值5的特征向量. 事实上$\alpha_1,\alpha_2,\alpha_3$是$\sigma$的特征向量在基$1,x,x^2$下的坐标,则$\sigma$在特征值$-1$下的特征子空间为$\spa(1-x^2,x-x^2)$,在特征值5下的特征子空间为$\spa(1+x+x^2)$.

    \item 正文中提到$f(A)$的特征值为$f(\lambda)$,其中$f$为多项式,$\lambda$是$A$的任意特征值,则$B$的特征值为$\mu_i=\lambda_i^3-5\lambda_i^2(i=1,2,3)$,即$\mu_1=-4,\mu_2=-6,\mu_3=-12$,再结合行列式等于特征值之积得到$|B|=\mu_1\mu_2\mu_3=-288$. 而$A+5E$的特征值分别为$\lambda_i+5(i=1,2,3)$,得到$|A+5E|=(1+5)(-1+5)(2+5)=168$. 而$|5E+P^{-1}AP|=|5P^{-1}P+P^{-1}AP|=|P^{-1}||5E+A||P|=|5E+A|=168$.

    \item 根据正文对伴随矩阵特征向量的讨论可知,伴随矩阵的特征向量和原矩阵一致,则$\alpha$也是$A$的特征向量,则由题意有
          \[\begin{pmatrix}
                  a & -1 & c \\ 5 & b & 3 \\ 1-c & 0 & -a
              \end{pmatrix}\begin{pmatrix}
                  -1 \\ -1 \\ 1
              \end{pmatrix}=\mu\begin{pmatrix}
                  -1 \\ -1 \\ 1
              \end{pmatrix},\]
          解线性方程组得$\mu=-1,b=-3,a=c$,则$A^*$对应于$\alpha$的特征值为$\dfrac{|A|}{\mu}=1$. 又由$|A|=\begin{vmatrix}
                  a & -1 & a \\ 5 & -3 & 3 \\ 1-a & 0 & -a
              \end{vmatrix}=a-3=-1$可得$a=c=2$.

    \item \label{ex:交换基础} 由题意有$AX=\lambda_0X$,则$A(BX)=B(AX)=\lambda_0(BX)$,可见$BX\in V_{\lambda_0}$.
\end{enumerate}

\centerline{\heiti B组}
\begin{enumerate}
    \item \begin{enumerate}
              \item 必要性:显然,因为任何一个向量都是数乘变换的特征向量;

              \item 充分性:由题意,任意向量都必然是$T$的特征向量,由正文中特征向量性质中最后的例题可知$T$是数乘变换.
          \end{enumerate}

    \item \begin{enumerate}
              \item $(\sigma/(\im\sigma))(v+\im\sigma)=\sigma(v)+\im\sigma,\enspace\forall v\in V$,由于$\sigma(v)\in\im\sigma$,则$\sigma(v)+\im\sigma=0+\im\sigma,\enspace\forall v\in V$,则$\sigma/(\im\sigma)$是零映射;

              \item 回忆单射的充要条件是核空间只有出发空间的零元,即$\sigma/(\ker \sigma)(v+U)=\sigma(v)+\ker\sigma=\ker\sigma$当且仅当$v\in\ker\sigma$,即$\sigma(v)\in\ker\sigma$当且仅当$v\in\ker\sigma$. 这一点与$\ker \sigma\cap\im \sigma=\{0\}$等价,因为
                    \begin{enumerate}
                        \item 必要性:$\forall v\in\ker \sigma\cap\im \sigma=\{0\}$,则$\exists u\in V$使得$v=\sigma(u)$,又$v\in\ker\sigma$,则$\sigma(u)\in\ker\sigma$,由已知$\sigma(v)\in\ker\sigma$当且仅当$v\in\ker\sigma$可得$u\in\ker\sigma$,即$v=\sigma(u)=0$;

                        \item 充分性:已知$\ker \sigma\cap\im \sigma=\{0\}$,若$\sigma(v)\in\ker\sigma$,由于$\sigma(v)\in\im\sigma$,则$\sigma(v)\in\ker\sigma\cap\im\sigma=\{0\}$,则$\sigma(v)=0$,即$v\in\ker\sigma$. 另一方面,若$v\in\ker\sigma$,则$\sigma(v)=0\in\ker\sigma$显然成立.
                    \end{enumerate}
          \end{enumerate}

    \item 设$\lambda$是$T/U$的特征值,则存在非零的$x+U(x\in v,x\notin U)$使得$(T/U)(x+U)=\lambda(x+U)$,即$Tx+U=\lambda x+U$,则$Tx-\lambda x\in U$. 若$\lambda$是$T\vert_U$的特征值,则得证;若不是,则$T\vert_U-\lambda I$可逆(则在有限维线性空间条件下是满射),因此存在$y\in U$使得$(T\vert_U-\lambda I)y=Tx-\lambda x$,即$Ty-\lambda y=Tx-\lambda x$,即$T(x-y)=\lambda(x-y)$,又$x\notin U$,$y\in U$,则$x-y\neq 0$,故$\lambda$是$T$的特征值.

    \item \begin{enumerate}
              \item 一维不变子空间就是特征子空间,由题意可知$T$有$n$个互异特征值,因此每个特征子空间都是一维不变子空间,因此$T$的所有一维不变子空间就是
                    \[\spa(e_1),\ldots,\spa(e_n)\]

              \item 事实上,我们很容易发现$\{0\},\spa(e_i),\spa(e_i,e_j),\ldots,\spa(e_1,\ldots,e_n)=V$都是$T$的不变子空间(因为每个$e_i$都是特征向量,所以很容易验证),这里一共有$C_n^0+C_n^1+\cdots+C_n^n=2^n$个不变子空间. 接下来说明$T$的不变子空间只有这$2^n$个. 设$W$是$T$的任一非零不变子空间,且$T$在$W$的基$\beta_1,\ldots,\beta_m$下的矩阵为$A$. 将$W$的基扩充为$V$的基$\beta_1,\ldots,\beta_m,\beta_{m+1},\ldots,\beta_n$,则$T$在$V$的基$\beta_1,\ldots,\beta_n$下的矩阵为
                    \[B=\begin{pmatrix}
                            A & C \\ O & D
                        \end{pmatrix},\]
                    则$T$的特征多项式据行列式运算性质可知为$|\lambda E-B|=\begin{vmatrix}
                            \lambda E-A & -C \\ O & \lambda E-D
                        \end{vmatrix}=|\lambda E-A||\lambda E-D|$,而$T\vert_W$的特征多项式为$|\lambda E-A|$整除$|\lambda E-B|$,由于$T$有$n$个互异特征值,这意味着$T\vert_W$也有$m$个互异特征值. 取其中任一特征值$\mu$,则存在$\beta$使得$T\vert_W\beta=\mu\beta$,即$T\beta=\mu\beta$,则$\mu$也是$T$的一个特征值,即$\mu$等于某个$\lambda_i$,即$T\beta=\lambda_i\beta$,于是$\beta$就是$\spa(e_i)$中的元素,因此某个$ke_i\in W(k\neq 0)$,故因为$W$是子空间,由运算封闭有$\spa(e_i)\subseteq W$. 从而我们也可以知道$T\vert_W$的$m$个互不相同的特征值只能是$\lambda_{i_1},\ldots,\lambda_{i_m}$,且$\spa(e_{i_1},\ldots,e_{i_m})\subseteq W$(因为每个$\spa(e_{i_k})$都在$W$中),由于$\dim W=m$,故$W=\spa(e_{i_1},\ldots,e_{i_m})$,结论得证.
          \end{enumerate}

    \item 本题探讨实数域上的二维线性空间,则不变子空间的维数只能为0,1,2,而维数为0和2对应$\{0\}$和$V$本身,这是无论$a$取何值时都一定有的,而一维不变子空间实际上就是单个特征向量张成的子空间,因此我们可以先求特征值:$|\lambda E-A|=\lambda^2+a-1$,则有如下讨论:
          \begin{enumerate}
              \item $a>1$,实数域上无特征值,因此所有不变子空间就是$\{0\}$和$V$本身;

              \item $a=1$,特征值为0,解得$A$的特征向量为$k(1,0)^T(k\in\mathbf{R})$,则$T$对应的不变子空间为$\spa(\alpha_1)$,当然不要忘记还有$\{0\}$和$V$本身;

              \item $a<1$,分别求解两个互异特征值的特征向量可以得到所有不变子空间为$\{0\}$,$\spa(\alpha_1+\alpha_2\sqrt{1-a})$,$\spa(\alpha_1-\alpha_2\sqrt{1-a})$和$V$本身.
          \end{enumerate}

    \item 由题意显然$A$和$B$不满秩(秩小于$n$),因此$A$和$B$有公共特征值0,且对应特征值0的特征子空间分别为$AX=0$和$BX=0$的解空间,它们的维数分别为$n-r(A)$和$n-r(B)$,二者维数和为$2n-(r(A)+r(B))>n$,因此它们的交集非零,即存在$X\neq 0$使得$AX=BX=0$,即$X$是$A$和$B$的共同的特征向量.

    \item 根据正文中的叙述,$A^2$的特征值就是$A$特征值的平方,此处不再赘述. 而$\displaystyle\sum_{i=1}^{n}\lambda_i^2$就是$A^2$特征值之和,即为$A^2$对角线元素之和,实际上简单验算就知道结果等于$\displaystyle\sum_{j=1}^{n}\sum_{k=1}^{n}a_{jk}a_{kj}$.

    \item 由题意有$(A-kE)X_1=0$,$(A-kE)X_2=lX_1$,$(A-kE)X_3=lX_2$. 设$m_1X_1+m_2X_2+m_3X_3=0$,两边左乘$A-kE$可得$m_2lX_1+m_3lX_2=0$,由于$l\neq 0$可知$m_2X_1+m_3X_2=0$. 两边再次左乘$A-kE$可得$lm_3X_1=0$,可知$m_3=0$,往前代入可知$m_2=m_1=0$,故$X_1,X_2,X_3$线性无关.
\end{enumerate}

\centerline{\heiti C组}
\begin{enumerate}
    \item 设$B$的特征值为$\lambda_1,\ldots,\lambda_n$,则$B$的特征多项式为$f(\lambda)=|\lambda E-B|=(\lambda-\lambda_1)\cdots(\lambda-\lambda_n)$,则$f(A)=(A-\lambda_1E)\cdots(A-\lambda_nE)$. $f(A)$可逆充要条件是$|f(A)|\neq 0$,即$|A-\lambda_iE|\neq 0,\enspace i=1,\ldots,n$,这等价于$A$的特征值不是$\lambda_1,\ldots,\lambda_n$,即$B$的特征值都不是$A$的特征值.

    \item 根据A组 \ref*{ex:交换基础} 题,$BX\in V_{\lambda_0}(A)$,故只需证存在$Z\neq 0$使得$Z\in V_{\lambda_0}(A)$且$BZ=\mu Z$,此时$Z$是$A$和$B$共同的特征向量. 设$X_1,\ldots,X_r$为$V_{\lambda_0}(A)$的基,由于$BX_i\in V_{\lambda_0}(A)$,则$BX_i$可以由$X_1,\ldots,X_r$线性表出,即$BX_i=(X_1,\ldots,X_r)\alpha_i$,其中$\alpha_i=(a_{i1},\ldots,a_{ir})^T$. 因此有
          \[B(X_1,\ldots,X_r)=(X_1,\ldots,X_r)(\alpha_1,\ldots,\alpha_r)=(X_1,\ldots,X_r)P,\]
          其中$r$阶矩阵$P=(\alpha_1,\ldots,\alpha_r)$在复数域上有特征值$\mu$,故存在$Y_0\neq 0$使得$PY_0=\mu Y_0$. 将上式两端右乘$Y_0$,得
          \[B(X_1,\ldots,X_r)Y_0=(X_1,\ldots,X_r)(PY_0)=(X_1,\ldots,X_r)(\mu Y_0)=\mu(X_1,\ldots,X_r)Y_0.\]
          令$Z_0=(X_1,\ldots,X_r)Y_0$,则$Z_0\neq 0$(因为坐标$Y_0\neq 0$)且$BZ_0=\mu Z_0$,即$Z_0$是$B$的特征向量. 又$Z_0=(X_1,\ldots,X_r)Y_0$,即是可由$V_{\lambda_0}$的基线性表示的,故$Z_0\in V_{\lambda_0}$,即$Z_0$是$A$的特征向量. 因此$Z_0$是$A$和$B$共同的特征向量.
\end{enumerate}

\clearpage
