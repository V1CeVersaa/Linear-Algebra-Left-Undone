\phantomsection
\section*{9 线性映射矩阵表示(III)}
\addcontentsline{toc}{section}{9 线性映射矩阵表示(III)}

\vspace{2ex}

\centerline{\heiti A组}
\begin{enumerate}
    \item $\alpha = (x_1,x_2,x_3)^{\mathrm{T}},\beta = (y_1,y_2,y_3)^{\mathrm{T}},\alpha^{\mathrm{T}}\beta=x_1y_1+x_2y_2+x_3y_3$,实际上是 $\alpha\beta^{\mathrm{T}}$ 对角线元素之和,故 $\alpha^{\mathrm{T}}\beta=-5$.
\end{enumerate}

\centerline{\heiti B组}
\begin{enumerate}
    \item $A=(a_{ij})_{n\times n}$,$A^\mathrm{T}A$ 对角线元素为 $\displaystyle\sum_{i=1}^na_{i1}^2,\sum_{i=1}^na_{i2}^2,\ldots,\sum_{i=1}^na_{in}^2$ 均为 0,故 $A$ 中所有元素均为 0,可得 $A=O$. 另一方法:由于 $r(A^\mathrm{T}A) = r(A)$,又 $r(A^\mathrm{T}A)=0$,所以 $r(A)=0$,从而 $A=O$.

    \item \begin{enumerate}
              \item 根据定义易证.

              \item 令 $E_{ij}$ 表示第 $i$ 行第 $j$ 列位置是 1,其余位置都是 0 的 $n$ 阶方阵,则 $V$ 的一组基是
                    \[E_{11},\ldots,E_{nn},E_{12}+E_{21},\ldots,E_{n-1,n}+E_{n,n-1}\]
                    故 $V$ 的维数是 $1+2+\cdots+n=\dfrac{n(n+1)}{2}$.
          \end{enumerate}

    \item 观察对称性有
          \[AA^{\mathrm{T}}=(a^2+b^2+c^2+d^2)E\]
          从而有 $A^{-1}=\dfrac{1}{a^2+b^2+c^2+d^2}A^{\mathrm{T}}$.

    \item 设 $E_{ij}$ 表示第 $i$ 行第 $j$ 列位置是 1,其余位置都是 0 的 $n$ 阶方阵.
          \begin{enumerate}
              \item $k=0$ 时,$W$ 为主对角线全为 0 的上三角矩阵全体,则 $B_1=\{E_{ij} \mid i>j\}$ 为 $W$ 的一组基,且 $\mathrm{dim}W=\dfrac{n(n-1)}{2}$.

              \item $k=1$ 时,$W$ 为对称矩阵全体,$\forall A\in W,A = \displaystyle\sum_{i=1}^n\sum_{j=1}^na_{ij}E_{ij}=\sum_{i<j}a_{ij}(E_{ij}+E_{ji})+\sum_{i=1}^na_{ii}E_{ii}$,而 $B_2=\{E_{ij}+E_{ji},E_{kk} \mid 1\leq i,j,k\leq n,i<j\}$ 线性无关,故 $B_2$ 是 $W$ 的一组基,$\mathrm{dim}W=\dfrac{n(n+1)}{2}$.

              \item $k=2$ 时,可得 $a_{ii}=2a_{ii},a_{ii}=0$,故 $\forall A\in W,A = \displaystyle\sum_{i=1}^n\sum_{j=1}^na_{ij}E_{ij}=\sum_{i<j}a_{ij}(E_{ij}+2E_{ji})$,而 $B_3=\{E_{ij}+2E_{ji} \mid i<j\}$ 线性无关,故 $B_3$ 是 $W$ 的一组基,$\mathrm{dim}W=\dfrac{n(n-1)}{2}$.
          \end{enumerate}
\end{enumerate}

\centerline{\heiti C组}
\begin{enumerate}
    \item
\end{enumerate}

\clearpage
